
\chapter{Preconditioning}
\label{chap:precond}

% Let us define the matrix of the linear system \eqref{eq:mhd_saddle} as $\mathcal{K}$.




The linear system \eqref{eq:mhd_saddle} is typically sparse and of large dimension, hence to efficiently solve for it we use a preconditioned iterative approach as proposed in \cite{li2010numerical}. We start by reviewing some preconditioning strategies for the incompressible Navier-Stokes and Maxwell subproblems in isolation. From these techniques we will then introduce and  numerically test  preconditioners for the full MHD system.

\section{Navier-Stokes equations}
\label{sec:NSprecond}


To start with, consider the steady state incompressible Navier-Stokes equations in isolation. Let
\begin{equation}
\label{eq:ns_coeff}
\mathcal{K}_{\rm NS}=
\begin{pmatrix}
F & B^T \\
B & 0
\end{pmatrix},
\end{equation}
be the discretised and linearised Navier-Stokes subproblem where $F~=~A~+~O$. Due to the convection term, $O$, this  system is non-symmetric and we will use GMRES to solve this subproblem \cite{saad1986gmres}. An excellent choice for a preconditioner for a saddle point system like this is to use a block diagonal or block triangular based preconditioner of the form
\begin{equation}
\label{eq:ns_pc_upper}
\mathcal{M}_{\rm NS} =
\begin{pmatrix}
F & B^T \\
0 & -S
\end{pmatrix},
\end{equation}
where the Schur complement $S$ is given by $S=B F^{-1} B^T$. It has been proved in \cite{murphy2000note} that for a suitable Krylov subspace method then the iterative scheme will converge in exactly $2$ iterations when using the block triangular preconditioner or $3$ iterations using a block diagonal where the $B^T$ is dropped from the $(1,2)$ block in $\mathcal{M}_{\rm NS}$.

In practice it is often too expensive to form and solve for the Schur complement, hence, a good approximation to it is needed. Two well known preconditioners for the incompressible Navier-Stokes equations are the Least Squares Commutator (LSC) and the Pressure Convection-Diffusion (PCD) preconditioners. Both can be found in \cite{elman2005finite} and we will just outline the procedure how these can be applied on the discrete level.

% Both methods (LSC and PCD)  start with the convection-diffusion operator associated with the velocity space $\uu{V}_h$ given by
% $$\mathcal{L} = -\nu \Delta +\uu{w} \cdot \nabla\, .$$
% As before $\uu{w}$ is the discrete velocity calculated at the previous non-linear iteration. Suppose that there is a corresponding operator defined in the pressure space
% $$\mathcal{L}_p = (-\nu \Delta +\uu{w} \cdot \nabla)_p\, .$$
% Consider the commutator of the convection-diffusion operator associated with the gradient operator
% \begin{equation} \label{eq:ContCommutator}
% \epsilon = (-\nu \Delta +\uu{w} \cdot \nabla)\nabla - \nabla (-\nu \Delta +\uu{w} \cdot \nabla)_p
% \end{equation}
% to be small. In fact, if $\uu{w}$ was constant then $\epsilon = 0$. We will be using \eqref{eq:ContCommutator} to derive both LSC and PCD.



\subsection{Pressure Convection-Diffusion (PCD)}
Before considering the PCD approach to approximate the Schur complement, we define the velocity mass matrix as $Q=(Q_{i,j})_{i,j=1}^{n_u} \in{\mathbb R}^{n_u \times n_u}$, where in terms of the basis $\{\uu{\psi_i}\}$
\begin{equation}
\label{eq:pressure_mass}
Q_{i,j}=
\int_\Omega\, \uu{\psi_j}\cdot \uu{\psi_i} \,d\uu{x}, \quad 1\leq i,j \leq n_u.
\end{equation}
In \cite[Chap. 8]{elman2005finite} the discrete commutator of the convection-diffusion operator associated with the gradient operation is introduced and given by
\begin{equation} \label{eq:DisCommutator}
    \epsilon_h = (Q^{-1}F)(Q^{-1}B^T)-(Q^{-1}B^T)(W^{-1}F_p)
\end{equation}
In this equation, $W=(W_{i,j})_{i,j=1}^{m_u} \in{\mathbb R}^{m_u \times m_u}$, $F_p=((F_{p})_{i,j})_{i,j=1}^{m_u} \in{\mathbb R}^{m_u \times m_u}$ and introduce $A_p=((A_{p})_{i,j})_{i,j=1}^{m_u} \in{\mathbb R}^{m_u \times m_u}$ (which will be used later) are mass matrix, convection diffusion operator and Laplacian matrix defined on the pressure space as:
\begin{equation}
\label{eq:PressureMass}
 \left. \begin{aligned}
W_{i,j}&= \int_\Omega\, \alpha_j \alpha_i \,dx, \ \ 1\leq i,j \leq m_u, \\
(F_{p})_{i,j}&= \nu \int_\Omega\, \grad \alpha_j \cdot \grad \alpha_i +(\uu{w} \cdot \grad \alpha_j)\alpha_i\,dx,\ \ 1\leq i,j \leq m_u, \\
(A_{p})_{i,j}&=  \int_\Omega\, \grad \alpha_j \cdot \grad \alpha_i \,dx, \ \ 1\leq i,j \leq m_u.
 \end{aligned}
 \right.
 \qquad \text{}
\end{equation}
These matricies are well-defined since our pressure spaces are continuous. Assuming that the commutator is small then pre and post multiplying \eqref{eq:DisCommutator} by $B F^{-1} Q$ and $F_p^{-1}W$, respectively, lets us separate the Schur complement to give
\begin{equation} \label{eq:SchurApprox}
    BF^{-1}B^T \approx B Q^{-1}B^T F_p^{-1} W.
\end{equation}

In general, $BF^{-1}B$ is both costly to form and usually dense, so it is impractical to use.  Our discretisation is inf-sup stable which means that there is spectral equivalency between $BQ^{-1}B^T$ and the pressure Laplacian, $A_p$, see \cite[Section 5.5.1]{elman2005finite}. Hence, the Schur complement can be approximated by:
$$S_{\rm PCD} =A_p F_p^{-1}W.$$
Applying the PCD preconditioner to the full Navier-Stokes system involves solving the system
\begin{equation} \nonumber
% \label{eq:matrix-system}
\left(
\begin{array}{cc}
F & B^T \\
0 & -A_p F_p^{-1}W
\end{array}
\right)
\,
\left(
\begin{array}{c}
x \\
y
\end{array}
\right) =
\left(
\begin{array}{c}a\\b
\end{array}
\right)
\end{equation}
at each Krylov iteration. This can be solved  efficiently by splitting it into the following two steps
\begin{itemize} \label{it:PCDsolve}
    \item[1.] Solve for $y$: $y = -W^{-1}F_p A_p^{-1}b$
    \item[2.] Solve for $x$: $x = F^{-1}(a-B^Ty).$
\end{itemize}
This means that we have one pressure Poisson solve ($A_p^{-1}$), one mass matrix solve ($W^{-1}$) and one convection-diffusion solve ($F^{-1}$) at each Krylov iteration. These solves will be done using direct solver unless specifically stated otherwise.

\subsection{Least Squares Commutator}

As for the derivation of the PCD preconditioner we start off with the discrete commutator of the convection-diffusion operator
\begin{equation} \nonumber
    \epsilon_h = (Q^{-1}F)(Q^{-1}B^T)-(Q^{-1}B^T)(W^{-1}F_p).
\end{equation}
Suppose that the $Q$-norm is defined by $\|v\|_{Q} = (Qv,v)^{\nicefrac{1}{2}}$. Then this time we minimise $\epsilon_h$ in the $Q$-norm to try to find an expression for $F_p$. The minimisation is given by
$$\min \|(Q^{-1}F)(Q^{-1}B^T)-(Q^{-1}B^T)(W^{-1}F_p) \|_Q.$$
Solving this optimisation problem, as shown in \cite{elman2005finite}, is equivalent to solving the following normal equations
$$W^{-1}BQ^{-1}B^TW^{-1}F_p = W^{-1}BQ^{-1} FQ^{-1}B^T.$$
This yields the following expression for $F_p$:
$$F_p = W(BQ^{-1}B^T)^{-1}(BQ^{-1} FQ^{-1}B^T).$$
By substitution this into expression \eqref{eq:SchurApprox} we obtain the LSC approximation to the Schur complement:
\begin{equation} \nonumber
    S = BF^{-1}B^T \approx S_{\rm LSC} = (B Q^{-1} B^T)(BQ^{-1}FQ^{-1}B^T)^{-1}(B Q^{-1} B^T).
\end{equation}
Therefore, applying the LSC preconditioner to the full Navier-Stokes system $\mathcal{K}_{\rm NS}$ in \eqref{eq:ns_coeff} involves solving for the matrix
\begin{equation} \nonumber
\left(
\begin{array}{cc}
F & B^T \\
0 & -S_{\rm LSC}
\end{array}
\right)
\,
\left(
\begin{array}{c}
x \\
y
\end{array}
\right) =
\left(
\begin{array}{c}a\\b
\end{array}
\right)
\end{equation}
at each Krylov iteration. Again, this can be split up into the following two steps:
\begin{itemize}
    \item[1.] Solve for $y$: $y = -(B Q^{-1} B^T)^{-1}(BQ^{-1}FQ^{-1}B^T)(B Q^{-1} B^T)^{-1}b$
    \item[2.] Solve for $x$: $x = F^{-1}(a-B^Ty).$
\end{itemize}
Hence, we have two pressure Poisson solves ($(B Q^{-1} B^T)^{-1}$) and one Convection-Diffusion solve ($F^{-1}$) at each Krylov iteration. In practice, we take the diagonal or lumped diagonal of $Q$ to form $B Q^{-1} B^T$. These solves, as in with the PCD preconditioner, will be done directly.


\section{Maxwell's equations}
\label{sec:MaxwellPrecond}

Next, consider the Maxwell subproblem
\begin{equation}
\label{eq:m_coeff}
\mathcal{K}_{\rm MX}=
\begin{pmatrix}
M & D^T \\
D & 0
\end{pmatrix}.
\end{equation}
As for the Navier-Stokes subproblem in Section \ref{sec:NSprecond}, we apply a block preconditioning strategy for $\mathcal{K}_{\rm MX}$ in  \eqref{eq:m_coeff}.

Recall that the $(1,1)$ block of $\mathcal{K}_{\rm MX}$ is the curl-curl operator, and hence the matrix $M$ is singular with nullity $m_b$ which corresponds to the discrete gradients. Therefore the usual Schur complement does not exist as it involves inverting $M$. To overcome this difficulty, we employ the approach in  \cite{golub2003solving,greif2006preconditioners} based on  augmentation. More precisely, we replace $M$ by $M+D^T\mathcal{W}^{-1}D$ where $\mathcal{W}\in {\mathbb R}^{m_b\times m_b}$ is a symmetric positive definite matrix, see \cite{golub2003solving,greif2006preconditioners} for more details. The addition of the matrix ($D^T\mathcal{W}^{-1}D$) removes the singularity of the $(1,1)$ block of $\mathcal{K}_{\rm MX}$ without changing the solution (since $Db = 0$). For the Maxwell subproblem the appropriate choice of $\mathcal{W}$ is the scalar Laplacian on $S_h$ defined as $L=(L_{i,j})_{i,j=1}^{m_b} \in{\mathbb R}^{m_b \times m_b}$ with
\begin{equation}
\label{eq:scalar_laplace}
L_{i,j}=\int_\Omega\,\nabla\beta_j\cdot\nabla\beta_i\,d\uu{x},
\end{equation}
see  \cite{greif2007preconditioners}. Therefore we will consider preconditioning the following augmented system:
\begin{equation}
\label{eq:AugmentMaxwell}
\bar{\mathcal{K}}_{\rm MX}=
\begin{pmatrix}
M + D^TL^{-1}D & D^T \\
D & 0
\end{pmatrix}.
\end{equation}

\subsection{An ideal preconditioner}


It has been shown in \cite{greif2007preconditioners} that an ideal preconditioner for $\bar{\mathcal{K}}_{\rm MX}$ in \eqref{eq:AugmentMaxwell} is the block diagonal matrix
\begin{equation}
\label{eq:maxwell_pc_ideal}
\mathcal{M}_{\rm iMX} =
\begin{pmatrix}
M+D^T L^{-1} D & 0 \\
0 & L
\end{pmatrix}.
\end{equation}
Applying \eqref{eq:maxwell_pc_ideal} as the preconditioner yields exactly two eigenvalues, $1$ and $-1$. Therefore using this matrix as a preconditioner means that MINRES will converge in two iterations  \cite{paige1975solution}. However, forming the matrix $M+D^T L^{-1} D$ is costly, hence, $\mathcal{M}_{\rm iMX}$  is  impractical  for large systems.


\subsection{A practical preconditioner}

A good approximation for $M+D^T L^{-1} D$ is required to make the ideal preconditioner, $\mathcal{M}_{\rm iMX}$, suitable in practise. It has been shown in \cite{greif2007preconditioners}  that $M+D^T L^{-1} D$ is spectrally equivalent to $M+X$  where $X=(X_{i,j})_{i,j=1}^{n_b}\in{\mathbb R}^{n_b \times n_b}$ is the mass matrix on the magnetic space and is defined as
\begin{equation}
\label{eq:magnetic_mass}
X_{i,j}=\int_\Omega\, \uu{\psi}_j\cdot\uu{\psi}_i\,d\uu{x}.
\end{equation}
Using this approximation leads to the practical preconditioner
\begin{equation}
\label{eq:maxwell_pc_X}
\mathcal{M}_{\rm MX} =
\begin{pmatrix}
N& 0 \\
0 & L
\end{pmatrix},
\end{equation}
where $N = M+X$.

\section{A preconditioner for the MHD problem}
\label{sec:MHDprecond}

Sections \ref{sec:NSprecond} and \ref{sec:MaxwellPrecond} looked briefly at the preconditioning strategies for the Navier-Stokes and Maxwell's equations. Using these techniques we will look at possible scalable preconditioners for the full MHD problem,
\begin{equation}
    {\mathcal K}_{\rm MH} = \left(
\begin{array}{cccc}
A+O & B^T & C^T & 0\\
B & 0 & 0 & 0\\
-C & 0 & M & D^T \\
0 & 0 & D & 0
\end{array}
\right).
\end{equation}
Using the Navier-Stokes and Maxwell subproblem preconditioners \eqref{eq:ns_pc_upper} and \eqref{eq:maxwell_pc_X} respectively, then we propose the following preconditioner for ${\mathcal K}_{\rm MH}$
\begin{equation}
\label{eq:mhd_pc_ls}
\mathcal{M}_{\rm MH} =
\left(
\begin{array}{cccc}
F & B^T & C^T & 0\\
0 & -S & 0 & 0 \\
-C & 0 & N & 0\\
0 & 0 & 0 & L
\end{array}
\right).
\end{equation}
Due to the coupling terms, $C$, the application of this precondioner is hard. To overcome this, we propose to invert  $\mathcal{M}_{\rm MH}$ by means of an inner preconditioner Krylov solver. The inner preconditioner is given by
\begin{equation}
\label{eq:mhd_pc_inner}
\mathcal{M}_{\rm innerMH} =
\left(
\begin{array}{cccc}
F & B^T & 0 & 0\\
0 & -{S} & 0 & 0 \\
0 & 0 & N & 0\\
0 & 0 & 0 & L
\end{array}
\right).
\end{equation}

\section{Preconditioners for MD and CD}

In Section \ref{sec:FEMdecouple} we introduced two decoupling schemes, namely Magnetic Decoupling (MD) and Complete Decoupling (CD). Using the results from sections \ref{sec:MaxwellPrecond} and \ref{sec:NSprecond} we will discuss the preconditioning approaches that we imply for these decoupling schemes.

\subsection{Magnetic decoupling}
\label{sec:MDprecond}

From section \ref{sec:FEMmd} the the matrix to be preconditioned is as follows:
\begin{equation}
   \mathcal{K}_{\rm MD} =
    \left(
    \begin{array}{cccc}
    F& B^T & 0 & 0\\
    B & 0 & 0 & 0 \\
    0 & 0 & M & D^T\\
    0 & 0 & D & 0
    \end{array}
    \right).
\end{equation}
Recall that removing the coupling terms completely decouples the system. This therefore enables us to use the optimal preconditioners for each of the subproblems separately and in parallel. Using the subproblem preconditioners \eqref{eq:maxwell_pc_X} and \eqref{eq:ns_pc_upper} then the optimal preconditioner for $\mathcal{K}_{\rm MD}$ is
\begin{equation}
\label{eq:mhd_pc_explicit}
\mathcal{M}_{\rm MD} =
\left(
\begin{array}{cccc}
F & B^T & 0 & 0\\
0 & -{S} & 0 & 0 \\
0 & 0 & N & 0\\
0 & 0 & 0 & L
\end{array}
\right).
\end{equation}


\subsection{Complete decoupling}
\label{sec:CDprecond}

To form an appropriate preconditioner for the CD iteration in \eqref{eq:matrix_CD} we first need to consider how to deal with the upper $(2,2)$ block matrix which corresponds to the discrete Stokes equations
\begin{equation}\nonumber
   \mathcal{K}_{\rm S} =
    \left(
    \begin{array}{cc}
    A& B^T \\
    B & 0
    \end{array}
    \right)
\end{equation}
As with the incompressible Navier-Stokes subproblem the idea for the Stokes preconditioner is again to approximate the Schur complement. The Schur complement associated with the Stokes system is
$$S_{\rm S} =  BA^{-1}B^T,$$
recall that the matrix $A$ is defined with the viscosity $\nu$ in section \ref{sec:variation}. It was shown in \cite{silvester1993fast,silvester1994fast} that the scaled pressure mass matrix, $\mbox{\small \(\frac{1}{\nu}\)} W$ defined in \eqref{eq:PressureMass}, is spectrally equivalent to the Schur complement (which is also a consequence of the inf-sup stability condition). Therefore the scalable Stokes preconditioner is
\begin{equation*}
\label{eq:mhd_pc_explicit2_1}
\begin{pmatrix}
A & 0 \\
0 & \mbox{\small \(\frac{1}{\nu}\)} W
\end{pmatrix}.
\end{equation*}
Using \eqref{eq:mhd_pc_explicit2} together with the Maxwell subproblem preconditioner  \eqref{eq:maxwell_pc_X} gives the preconditioner
\begin{equation}
\label{eq:mhd_pc_explicit2}
\mathcal{M}_{\rm CD} =
\left(
\begin{array}{cccc}
A & 0 & 0 & 0\\
0 & \mbox{\small \(\frac{1}{\nu}\)} W & 0 & 0 \\
0 & 0 & N & 0\\
0 & 0 & 0 & L
\end{array}
\right).
\end{equation}
The biggest advantage of this decoupling approach is that the matrix system is now symmetric. This means that the appropriate choice for the Krylov subspace method is MINRES for each subproblem.
