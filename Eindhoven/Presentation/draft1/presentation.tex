\documentclass[handout]{beamer}
\usepackage{subfigure}
\usepackage{graphicx}
\usepackage{amsfonts}
\usepackage{amsmath}
\usepackage{amsthm}
\usepackage{wrapfig}
\usepackage{amssymb}
\usepackage{nicefrac}
\makeatletter
\def\handoutsmode{handoutsmode}
\def\notesmode{notesmode}
\ifx\modetype\handoutsmode
  % Handouts mode: supress overlays
  \gdef\beamer@currentmode{handout}
\else\relax\fi
\ifx\modetype\notesmode
  % Notes mode: show notes and suppress overlays
  \gdef\beamer@currentmode{handout}
  \setbeameroption{show notes}
\else\relax\fi
\makeatother

% -- BEAMER PACKAGES & OPTIONS --
 \setbeamersize{text margin  left=.5cm}   % Default 1cm
 \setbeamersize{text margin right=.5cm}   % Default 1cm
 \usefonttheme[onlymath]{serif}           % Serif math font
 \setbeamertemplate{itemize items}[circle]
 \setbeamertemplate{section in toc}[sections numbered]
 \setbeamertemplate{navigation symbols}{} % No navigation symbols
%\setbeamercovered{invisible}             % Shadowed/invisible overlays
%\setbeameroption{show notes}
 \setbeamercolor{frametitle}{fg=black}
 \setbeamerfont{frametitle}{size=\large,series=\bfseries}
 \setbeamertemplate{frametitle}
 {
   \begin{centering}
     \insertframetitle\par
   \end{centering}
 }
% \setbeamertemplate{footline}[page number]  % Simple n/N footer
 \setbeamertemplate{footline}[text line]{%
   \vbox{%
%     \tinycolouredline{black}{\color{white}\bf%
       \insertpart\hfill%
       \insertpartnumber--\insertframenumber%
       \smallskip
 }}

% -- PACKAGES --
\usepackage{amsmath,amssymb,amsfonts}
% \usepackage{bbding}
% \usepackage{pstricks}

% \usepackage[misc]{ifsym}
\usepackage{braket}
\usepackage{boxedminipage}
\usepackage{overpic}
% -- SLIDE MACROS --
\def\bc{\begin{center}}
%\def\be{\begin{enumerate}}
\def\bi{\begin{itemize}}
\def\bs{\begin{small}}
\def\ec{\end{center}}
%\def\ee{\end{enumerate}}
\def\ei{\end{itemize}}
\def\es{\end{small}}


% -- ONE-LINE SLIDES --
\newcommand{\TOPIC}[1]{
  \frame{LARGE\textbf{%
      \begin{center}
        \gr{#1}
      \end{center}}}}

% -- BOXED EQUATIONS --
%\usepackage{empheq}
\newenvironment{boxedeq}%
  {\begin{empheq}[box=\fbox]{align}}
  {\end{empheq}}

% -- MISC --
\newcommand{\com}[1]{\texttt{#1}}
\newcommand{\DIV}{\ensuremath{\mathop{\mathbf{DIV}}}}
\newcommand{\GRAD}{\ensuremath{\mathop{\mathbf{GRAD}}}}
\newcommand{\CURL}{\ensuremath{\mathop{\mathbf{CURL}}}}
\newcommand{\CURLt}{\ensuremath{\mathop{\overline{\mathbf{CURL}}}}}
\newcommand{\nullspace}{\ensuremath{\mathop{\mathrm{null}}}}
\newcommand{\BALL}{{\color{structure}$\bullet \;$}}
\newcommand{\eq}{\ =}
\newcommand{\plus}{\ +}
\newcommand{\footbar}{\hspace*{-2.5ex}\rule{2in}{.2pt}\\}
\newcommand{\topline}{\hrulefill}
\newcommand{\botline}{\vspace*{-1ex}\hrulefill}
\renewcommand{\emph}[1]{\textbf{#1}}
\newcommand{\mcol}[3]{\multicolumn{#1}{#2}{#3}}
\newcommand{\assign}{\ensuremath{\leftarrow}}
\newcommand{\textbox}[2]{%
% \renewcommand{}[1]{{#1}}
\newcommand{\nedelec}{N\'{e}d\'{e}lec }

  \begin{tabular}{@{}#1@{}}%
    #2
  \end{tabular}}
\def\paper#1{\textcolor{darkyellow}{[#1]}}
\newcommand{\tss}[1]{{\scriptscriptstyle #1}}
\renewcommand{\Re}{\ensuremath{\mathbf{R}}}

\newcommand{\squishlist}{
   \begin{list}{$\bullet$}
    { \setlength{\itemsep}{0pt}      \setlength{\parsep}{3pt}
      \setlength{\topsep}{3pt}       \setlength{\partopsep}{0pt}
      \setlength{\leftmargin}{1.5em} \setlength{\labelwidth}{1em}
      \setlength{\labelsep}{0.5em} } }

\newcommand{\barelist}{
   \begin{list}{}
    { \setlength{\itemsep}{0pt}      \setlength{\parsep}{3pt}
      \setlength{\topsep}{3pt}       \setlength{\partopsep}{0pt}
      \setlength{\leftmargin}{0em} \setlength{\labelwidth}{1em}
      \setlength{\labelsep}{0.5em} } }

\newcommand{\squishlisttwo}{
   \begin{list}{$\bullet$}
    { \setlength{\itemsep}{0pt}    \setlength{\parsep}{0pt}
      \setlength{\topsep}{0pt}     \setlength{\partopsep}{0pt}
      \setlength{\leftmargin}{2em} \setlength{\labelwidth}{1.5em}
      \setlength{\labelsep}{0.5em} } }

\newcommand{\squishend}{
    \end{list}  }


% -- COLORS --
\definecolor{darkgreen}{rgb}{0,0.5,0}
\definecolor{darkyellow}{rgb}{.8,.6,.04}
\newcommand{\gr}[1]{\textcolor{darkgreen} {#1}}
\newcommand{\wh}[1]{\textcolor{white}     {#1}}
\newcommand{\dy}[1]{\textcolor{darkyellow}{#1}}
\newcommand{\yb}[1]{\colorbox {yellow}    {#1}}
\newcommand{\re}[1]{{\textcolor{red}       {#1}}}
\newcommand{\bl}[1]{{\textcolor{blue}{#1}}}

\newcommand{\RE}[1]{{\bf\textcolor{red}       {#1}}}
\newcommand{\GR}[1]{{\bf\textcolor{darkgreen} {#1}}}
\newcommand{\DY}[1]{{\bf\textcolor{darkyellow}{#1}}}
\newcommand{\BL}[1]{{\bf\textcolor{blue}{#1}}}
\newcommand{\ssec}[1]{{\bf #1}}
\newcommand{\rsec}[1]{{\bf\color{red}       #1}}
\newcommand{\bsec}[1]{{\bf\color{blue}      #1}}
\newcommand{\gsec}[1]{{\bf\color{darkgreen} #1}}
\newcommand{\dom}{\mbox{\sf dom}}

\newcommand{\curl}{\ensuremath{\nabla\times\,}}
\renewcommand{\div}{\nabla\cdot\,}
\newcommand{\grad}{\ensuremath{\nabla}}

% -- SPACING --
\def\TabS  {\\ \hspace*{12pt}}
\def\TabSS {\\ \hspace*{30pt}}
\def\TabSSS{\\ \hspace*{45pt}}

\def\fourth{{\textstyle{\frac{1}{4}}}}
\newcommand{\rock}{\mcol{1}{l}{\bf Rock}}
\renewcommand{\paper}{\mcol{1}{l}{\bf Paper}}
\newcommand{\scissors}{\mcol{1}{l}{\bf Scissors}}
\newcommand{\Px}{{\bf X}}
\newcommand{\Py}{{\bf Y}}
\renewcommand{\div}{\nabla\cdot\,}
%\newcommand{Lt}{\hbox{\bf Left}}
\newcommand{\Rt}{\hbox{\bf Right}}
\newcommand{\iLt}{\invisible{Lt}}
\newcommand{\iRt}{\invisible{\Rt}}
\usetheme{CambridgeUS}
\useinnertheme{rectangles}
\useoutertheme{infolines}
\usecolortheme{whale}
\definecolor{mygreen}{cmyk}{0.82,0.11,1,0.25}
\setbeamercolor{structure}{fg=mygreen}
\setbeamercolor{frametitle}{fg=black,bg=mygreen}
\usefonttheme[onlylarge]{structuresmallcapsserif}
\usefonttheme[onlysmall]{structurebold}
%\setbeamerfont{title}{shape=\itshape,family=\rmfamily}
%\setbeamercolor{title}{fg=black!80!black, bg=white!70!blue}
%\useoutertheme{miniframes}
%\usecolortheme{rose}
\beamertemplatetransparentcovereddynamic


\title{MHD}
\author[]{Chen Greif, Dan Li, Dominik Sch\"{o}tzau, \underline{Michael Wathen}}
\institute[]{\Big{Eindhoven precond}}
\date[]{$17^{\tiny{\mbox{th}}}$ June 2015 \\ Preconditioning 2015 \\ Eindhoven}
\begin{document}

\begin{frame}
\institute[]{The University of British Columbia}

\title{Block Preconditioners for an Incompressible Magnetohydrodynamics Problem}
\titlepage
\author{Michael Wathen}

\author{Michael Wathen}

\title{MHD}
\author{Michael Wathen}

\end{frame}
\title{MHD}
% \section{Introduction}

% \begin{frame}

% \frametitle{Continuous and Discrete Maxwell's equations}
% \begin{tabular}{lrrrr}
% \hline
% {} &  Grid size &      DoF &  $\#$ iters &  Soln Time \\
% \hline
% 0 &       $   2^3$ &       81 &        1 &   4.25e-04 \\
% 1 &       $   4^3$ &      375 &        3 &   6.03e-04 \\
% 2 &       $   8^3$ &     2187 &        5 &   2.53e-03 \\
% 3 &      $  16^3$ &    14739 &        5 &   1.96e-02 \\
% 4 &      $  32^3$ &   107811 &        6 &   2.24e-01 \\
% 5 &      $  64^3$ &   823875 &        6 &   2.28e+00 \\
% 6 &     $ 128^3$ &  6440067 &        6 &   2.09e+01 \\
% \hline
% \end{tabular}

% \end{frame}


% \section{Overview}
% \begin{frame}
% \begin{center}
%   Overview
% \end{center}

% \begin{itemize}
%   \item Problem background: Setup, ...
%   \item
% \end{itemize}

% \end{frame}

\section{Problem background}
\begin{frame}{Problem background}

\begin{itemize}
  \item MHD models electrically conductive fluids (such as liquid metals, plasma, salt water, etc) in an electic field
  \pause
  \item Applications: electromagnetic pumping, aluminium electrolysis, the Earth's molten core and solar flares
  \pause
  \item MHD models couple electromagnetism (governed by Maxwell's equations) and fluid dynamics (governed by the Navier-Stokes equations)
  \pause
  \item Movement of the conductive material that induces and modifies any existing electromagnetic field
  \pause
  \item Magnetic and electric fields generate a mechanical force on the fluid
\end{itemize}

\end{frame}

\subsection{Navier-Stokes Equations} % (fold)

\begin{frame}
\frametitle{Steady-state Navier-Stokes equations}
Incompressible Navier-Stokes Equations:

\begin{subequations}\nonumber
  \re{\begin{alignat}2
    - \nu  \, \Delta{u}+({u} \cdot \nabla){u} +\nabla p &= {f} & \qquad &\mbox{in $\Omega$},\\[.1cm]
    \nabla\cdot{u} &= 0 & \qquad &\mbox{in $\Omega$},\\[.1cm]
    u &= u_D & \qquad &\mbox{on $\partial \Omega$}
    \end{alignat}}
\end{subequations}
where $\re{u}$ is the fluids velocity; $\re{p}$ is the fluids pressure; $\re{f}$ is the body force acting on the fluid and $\re{\nu}$ the kinematic viscosity.


\end{frame}


\begin{frame}{Steady-state Navier-Stokes equations}

Corresponding linear system:
$$\re{\mathcal{K}_{\rm NS}=\begin{pmatrix}
A+0 & B^T \\
B & 0
\end{pmatrix}
\begin{pmatrix}
u\\p
\end{pmatrix}=
\begin{pmatrix}
f \\ 0
\end{pmatrix}
}$$
where $\re{A\in\mathbb{R}^{n_u\times n_u}}$ is the discrete Laplacian; $\re{O\in\mathbb{R}^{n_u\times n_u}}$ is the convection operator and $\re{B\in\mathbb{R}^{m_u\times n_u}}$ is a discrete divergence operator with full row rank.

\bl{Note}: due to convection term the linear system is non-symmetric

\vspace{2mm}

For an extensive discussion of preconditioners we refer to \gr{Elman, Silvester and Wathen 2005/2014}.

\end{frame}



\subsection{Maxwell's Equations} % (fold)
\begin{frame}{Time-Harmonic Maxwell in mixed form}
Maxwell operator in mixed form:
\begin{subequations}\nonumber
  \re{\begin{alignat}2
    \nabla\times( \nabla\times {b}) -k^2 b +\grad r &= {g} & \qquad &\mbox{in $\Omega$},\\[.1cm]
    \nabla\cdot{b} &= 0 & \qquad &\mbox{in $\Omega$},\\[.1cm]
    b \times n &= b_D & \qquad &\mbox{on $\partial \Omega$},\\[.1cm]
    r &= 0& \qquad &\mbox{on $\partial \Omega$},
    \end{alignat}}\\[.1cm]
\end{subequations}
where $\re{b}$ is the magnetic vector field, $\re{r}$ is the scalar multiplier and $\re{k}$ is the wave number.

\vspace{2mm}

Note: for the MHD system $\re{k = 0}$

\end{frame}

\begin{frame}
Discretised and linearised Incompressible Navier-Stokes system
\begin{equation}
\nonumber
\re{\begin{pmatrix}
M-k^2X & D^T \\
D & 0
\end{pmatrix}
\begin{pmatrix}
b \\
r
\end{pmatrix}
=
\begin{pmatrix}
g \\
0
\end{pmatrix}},
\end{equation}
where $\re{M\in\mathbb{R}^{n_b\times n_b}}$ is the discrete curl-curl operator; $\re{X\in\mathbb{R}^{n_b\times n_b}}$ the discrete mass matrix; $\re{D\in\mathbb{R}^{m_b\times n_b}}$ the discrete divergence operator

\vspace{2mm}

\bl{Note:} $\re{M}$ is semidefinate with nulity $\re{m_b}$.

\vspace{2mm}

A significant amount of literature on time-harmonic Maxwell: notably for us, work of \gr{Hiptmair 1999} and \gr{Hiptmair \& Xu 2007} for solving shifted curl-curl equations in a fully scalable fashion.



\end{frame}


\section{MHD}
\begin{frame}{MHD model: coupled Navier-Stokes and Maxwell's equations}

\begin{subequations} \nonumber
% \label{eq:mhd}
\re{\begin{alignat}2
% \label{eq:mhd1}
 - \nu  \, \Delta{u} + ({u} \cdot \nabla)
{u}+\nabla p  {\only<2>{\color{blue}}- \kappa\,
(\nabla\times{b})\times{b}} &= {f} & \qquad &\mbox{in $\Omega$},\\[.1cm]
% \label{eq:mhd2}
\nabla\cdot{u} &= 0 & \qquad &\mbox{in $\Omega$},\\[.1cm]
% \label{eq:mhd3}
\kappa\nu_m  \, \nabla\times( \nabla\times {b})
+ \nabla r
{\only<2>{\color{blue}}- \kappa \, \nabla\times({u}\times {b})}  &= {g} & \qquad &\mbox{in $\Omega$},\\[.1cm]
% \label{eq:mhd4}
 \nabla\cdot{b} &= 0 & \qquad &\mbox{in $\Omega$},
\end{alignat}}
\end{subequations}
with appropriate boundary conditions.
\pause
\begin{itemize}
  \item ${\color{blue}(\nabla\times{b})\times{b}}$:  Lorentz force accelerates the fluid particles in the direction normal to
 the electric and magnetic fields.

  \item ${\color{blue} \nabla\times({u}\times {b})}$: electromotive force modifying the magnetic field
\end{itemize}
\end{frame}

\section{Discretisation}
\begin{frame}{Discretisation}

\begin{itemize}
  \item Finite element discretisation based on the formulation in \gr{Sch{\"o}tzau 2004}
  \item Fluid variables: lowest order Taylor-Hood (${\mathcal P_2}/{\mathcal P_1}$)
  \item Magnetic variables: mixed {N\'{e}d\'{e}lec} element approximation
  \item {N\'{e}d\'{e}lec} elements capture solutions correctly on non-convex domains
\end{itemize}



\end{frame}

\begin{frame}
Discretised and linearised MHD model:
\begin{equation}
\nonumber %\mathcal{K} x \equiv
\re{\left(
\begin{array}{cccc}
A+O(u) & B^T & C(b)^T & 0\\
B & 0 & 0 & 0 \\
-C(b) & 0 & M & D^T\\
0 & 0 & D & 0
\end{array}
\right)
\,
\left(
\begin{array}{c}
\delta u\\
\delta p\\
\delta b\\
\delta r
\end{array}
\right)  =
\begin{pmatrix}
r_u \\
r_p\\
r_b\\
r_r
\end{pmatrix},}
\end{equation}
with
\begin{equation}\nonumber
\re{\begin{array}{rl}
r_u &= f- Au -O(u) u - C(b)^T b- B^T p,\\
r_p &=-B u,\\
r_b &=g-Mu+C(b)b-D^T r,\\
r_r &=-D b.
\end{array}}
\end{equation}
  $\re{A}$:~discrete Laplacian operator, $\re{O}$:~discrete convection operator, $\re{B}$:~discrete divergence operator, $\re{M}$:~discrete curl-curl operator, $\re{C}$:~coupling terms, $\re{D}$:~discrete divergence operator.
  \end{frame}


\begin{frame}
\begin{center}
  {\Large Decoupling schemes}
\end{center}

Magnetic Decoupling (MD):
  $$\re{\mathcal{K}_{\rm MD}=\left(
\begin{array}{cc|cc}
A+O(u) & B^T & 0 & 0\\
B & 0 & 0 & 0 \\
\hline
0 & 0 & M & D^T\\
0 & 0 & D & 0
\end{array}\right)}
$$

Complete Decoupling (CD):
$$\re{\mathcal{K}_{\rm CD}=\left(
\begin{array}{cc|cc}
A & B^T & 0 & 0\\
B & 0 & 0 & 0 \\
\hline
0 & 0 & M & D^T\\
0 & 0 & D & 0
\end{array}
\right)}
$$
\end{frame}

\section{Preconditioning}

\begin{frame}{Linear solver and Preconditioning}
Consider
$$\re{Ax=b},$$
to iteratively solve:
$$\mbox{find }\re{x_k \in x_0+{\rm span}\{r_0,Ar_0, \ldots ,A^{k-1}r_0\}}$$
where $\re{r_0 = b-Ax_0}$ and $\re{x_0}$ is the initial guess.

\vspace{5mm}

Key for success: preconditioning
\begin{itemize}
    \item[1.] the preconditioner $\re{P}$ approximates $\re{A}$
    \item[2.]  $\re{P}$ easy to solve for than $\re{A}$
\end{itemize}
Want eigenvalues of $\re{P  ^{-1}A}$  to be clusters

\end{frame}



\begin{frame}{Ideal preconditioning}

Non-singular $(1,1)$ block
\begin{equation}\nonumber
\re{\mathcal{K} = \begin{pmatrix}
F & B^T \\
B & 0
\end{pmatrix}; \quad
\mathcal{P}=\begin{pmatrix}
F & B^T\\
0 & B F^{-1} B^T
\end{pmatrix}}
\end{equation}
\gr{Murphy, Golub \& Wathen 2000} showed exactly two eigenvalues: $\pm 1$ %and $\nicefrac{1}{2}\pm \nicefrac{\sqrt{5}}{2}$

\vspace{5mm}
\pause
{Singular} $(1,1)$ block
\begin{equation}\nonumber
\re{\mathcal{K} = \begin{pmatrix}
F & B^T \\
B & 0
\end{pmatrix}; \quad
\mathcal{P}=\begin{pmatrix}
F+B^T W^{-1} B & 0 \\
0 & W
\end{pmatrix}}, \ \mbox{where $\re{W}$ is SPD}
\end{equation}
\gr{Greif \& Sch{\"o}tzau 2006} showed exactly two eigenvalues: $\pm 1$

\end{frame}

\begin{frame}{Navier-Stokes subproblem}

$$\re{\mathcal{K}_{\rm NS}=\begin{pmatrix}
F & B^T \\
B & 0
\end{pmatrix}}$$
where $\re{F=A+O}$ is the discrete convection diffusion operator. Shown in \gr{Elman, Silvester \& Wathen 2005/2014} that
$$\re{\mathcal{P}_{\rm NS}=\left(\begin{array}{cc}
F & B^T \\
0 & S
\end{array}\right)}, \quad \re{S =A_p F_p^{-1}Q_p}$$
is a good approximation to the Schur complement preconditioner. $\re{A_p}$:~pressure space Laplacian, $\re{F_p}$:~pressure space convection-diffusion operator, $\re{Q_p}$:~pressure space mass matrix
\end{frame}


\begin{frame}{Maxwell subproblem}
$$\re{\mathcal{K}_{\rm NS}=\begin{pmatrix}
M & B^T \\
B & 0
\end{pmatrix}}$$
{\textit Note:} M is highly rank defficient.

\gr{Greif \& Sch{\"o}tzau 2007} shows that $\re{L}$ (scalar Laplacian) is the appropriate choice for $\re{W}$
$$\re{\mathcal{P}_{\rm iM}=\left(\begin{array}{cc}
M+B^T L^{-1} B & 0 \\
0 & L
\end{array}\right)}$$
\pause
Practical preconditioner:
$$\re{\mathcal{P}_{\rm M}=\left(\begin{array}{cc}
M+X & 0 \\
0 & L
\end{array}\right)}$$
where $\re{X}$ vector mass matrix is spectrally equivalent to $\re{B^T L^{-1} B}$
\end{frame}


\begin{frame}{MHD problem}
  Combining the Navier-Stokes and Maxwell preconditioners
  $$\re{\mathcal{P}_{\rm MH} =
  \left(
  \begin{array}{cccc}
  F  & B^T & C^T & 0\\
  0 & -{S} & 0 & 0 \\
  -C & 0 & M+X & 0\\
  0 & 0 & 0 & L
  \end{array}
  \right)}$$

\vspace{2mm}

\bl{Note:} $\re{{\mathcal{P}_{\rm MH}}}$ remains impractical due to coupling terms. Dual Schur complement approximation for velocity-magnetic unknowns

$$\re{\mathcal{P}_{\rm schurMH} =
\left(
\begin{array}{cccc}
F  + N_C & B^T & 0 & 0\\
0 & -{S} & 0 & 0 \\
0 & 0 & M+X & 0\\
0 & 0 & 0 & L
\end{array}
\right)}$$
where $\re{N_C = C^T (M + X)^{-1}C}$
\end{frame}


\begin{frame}{Approximation for $\re{N_C}$}

\end{frame}

\begin{frame}{Summary of decoupling scheme preconditioners}
  \begin{table}[h!]
  \begin{center}
  \begin{tabular}{|c|c|c|}
  \hline
    Iteration & Coefficient & Preconditioner \\
    scheme & matrix & \\
  \hline
%   \rule{0pt}{12pt}(P) & $ \left(
% \begin{array}{cccc}
% A+O & B^T & C^T & 0\\
% B & 0 & 0 & 0\\
% -C & 0 & M & D^T \\
% 0 & 0 & D & 0
% \end{array}
% \right)$ & $\left(
% \begin{array}{cccc}
% F & B^T & C^T & 0\\
% 0 & -{S} & 0 & 0 \\
% -C & 0 & N & 0\\
% 0 & 0 & 0 & L
% \end{array}
% \right)$ \\[0.1cm]
%   \hline
  \rule{0pt}{20pt}(MD) & $\re{ \left(
      \begin{array}{cc|cc}
      F& B^T & 0 & 0\\
      B & 0 & 0 & 0 \\
      \hline
      0 & 0 & M & D^T\\
      0 & 0 & D & 0
      \end{array}
      \right)}$ & $\re{\left(
      \begin{array}{cc|cc}
      F & B^T & 0 & 0\\
      0 & -{S} & 0 & 0 \\
      \hline
      0 & 0 & M+X & 0\\
      0 & 0 & 0 & L
      \end{array}
      \right)}$  \\[0.1cm]
  \hline
  \rule{0pt}{20pt}(CD) & $\re{\left(
  \begin{array}{cc|cc}
  A & B^T & 0 & 0\\
  B & 0 & 0 & 0 \\
  \hline
  0 & 0 & M & D^T\\
  0 & 0 & D & 0
  \end{array}
  \right)}$ &$\re{\left(\begin{array}{cc|cc}
  A & 0 & 0 & 0\\
  0 & \mbox{\small \(\frac{1}{\nu}\)} Q_p & 0 & 0 \\
  \hline
  0 & 0 & M+X & 0\\
  0 & 0 & 0 & L
  \end{array}
  \right)}$ \\[0.1cm]
  \hline
  \end{tabular}
  \caption{Summary of coefficient matrices and corresponding preconditioners for each the decoupling scheme}
  \label{tab:SummaryTable}
  \end{center}
  \end{table}
\end{frame}

\section{Numerical results}

\begin{frame}{Numerical software used}

\begin{itemize}
  \item Finite element software \re{\tt FEniCS}: core libraries are the problem-solving interface \re{\tt DOLFIN},  the compiler for finite element variational forms \re{\tt FFC}, the finite element  tabulator \re{\tt FIAT} for creating finite element function spaces, the just-in-time compiler \re{\tt Instant},  the code generator \re{\tt UFC} and  the form language \re{\tt UFL}.
  \item Linear algebra software: \re{\tt HYPRE} as a multigrid solver and the sparse direct solvers  \re{\tt UMFPACK}, \re{\tt PASTIX}, \re{\tt SuperLU} and \re{\tt MUMPS}
\end{itemize}
\end{frame}



% \begin{frame}
% \begin{equation} \nonumber
%     \begin{aligned}
%       \re{  -\nu \Delta \vec u+ \nabla p} \ & \re{= \vec f} \\
%          \re{ \div \vec u   } \ & \re{= 0 }\\
%         % \vec u &= \vec 0 \ \ \ \mbox{on } \partial\Omega}
%     \end{aligned} \ \ \ \ \  \mbox{in } \Omega \hspace{24mm}
% \end{equation}
% \vspace{-3mm}
% \begin{equation} \nonumber
% \mbox{\hspace{-1.5mm}} \re{\vec u = \vec g} \ \ \ \ \ \ \mbox{on } \partial\Omega
% \end{equation}

% Viscosity $ \re{\nu}$, velocity $ \re{\vec u}$ and pressure $ \re{p}$

% \end{frame}

% \begin{frame}

% \frametitle{Stokes}

% $$ \re{\mathcal{K} = \begin{bmatrix}
% A & B^{\mbox{\tiny{T}}}\\
% B & 0\\
% \end{bmatrix} \hspace{15mm}
% \mathcal{M} = \begin{bmatrix}
% A & 0\\
%  0& M\\
% \end{bmatrix}}
% $$

% \end{frame}


\section{Future work}
\begin{frame}

Future work:
\begin{itemize}
\item Scalable inner solvers
\item Release code on a public repository
\item Parallelisation of the code
\item Robustness with respect to kinematic viscosity
\item Other non-linear solvers
\item Different mixed finite element discretisations
\end{itemize}
\end{frame}


\end{document}